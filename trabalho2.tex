\documentclass{article}
\usepackage{amsthm}
\usepackage{amsmath}
\usepackage{amsfonts}
\usepackage{tikz}
\usepackage{amssymb}

\newtheorem*{definition}{Definição}
\newtheorem*{theorem}{Teorema}
\newtheorem*{exemple}{Exemplo}
\newtheorem*{atencao}{Atenção}
\newtheorem*{proposicao}{Proposição}
\newtheorem*{ex}{Exercício}
\usepackage{mathtools}
\usepackage{xfrac}
\DeclarePairedDelimiter\abs{\lvert}{\rvert}%
\DeclareMathOperator{\Ima}{Im}

\setlength{\textwidth}{450pt}
\setlength{\marginparwidth}{0pt}
\setlength{\marginparsep}{0pt}
\usepackage[left=2.5cm, right=2.5cm]{geometry}

\title{Trabalho 2 \\ \large Grupos e Corpos}
\author{Yuri Kosfeld}
\date{Junho 2025}

\begin{document}

\maketitle

\begin{ex}[4.1.8]
    a) Poderiámos provar utilizando Eisenstein, mas vamos mostrar por um processo analogo a um exemplo dessa seção.
    Informalmente, sabemos que uma raiz de $x^2 -3$ é $\sqrt{3}$, assim sabemos que o polinomio pode ser reescrito como 
    $x^2 -3 = (x - \sqrt{3})(x + \sqrt{3})$. Então se $\sqrt{3} \in \mathbb{Q}(\sqrt{2})$, o polinomio é redutivel.
    Suponha então que $\sqrt{3} \in \mathbb{Q}(\sqrt{2})$. Logo $\exists a, b \in \mathbb{Q}$ tais que $\sqrt{3} = a + b\sqrt{2}$.
    \begin{align*}
        (\sqrt{3})^2 &= (a + b\sqrt{2})^2 \\
        3 &= a^2 + 2ab\sqrt{2} + 2b^2 \\
        &= (a^2 + 2b^2) + (2ab)\sqrt{2}\\
    \end{align*}
    Assim temos, $a^2 + 2b^2 = 3$ e $2ab = 0$.
    Como $2ab = 0$, então ou $a = 0$ ou $b = 0$.
    \begin{itemize}
        \item Se $a = 0$, então $2b^2 = 3 \Rightarrow b = \sqrt{3/2}$ e logo $b \notin \mathbb{Q}$, absurdo.
        \item O caso $b = 0$ é analago ao anterior, também chegando em um absurdo. 
    \end{itemize}
    Logo $\sqrt{3} \notin \mathbb{Q}(\sqrt{2})$ e assim o polinomio é irredutivel.
    b) \[x^4 -10x^2 + 1 = (x^2 -2x\sqrt{3} + 1)(x^2 +2x\sqrt{3} + 1)\]
    $(x^2 -2x\sqrt{3} + 1), (x^2 +2x\sqrt{3} + 1)$ são redutiveis em $\mathbb{Q} [x]$.
\end{ex}

\begin{ex}[4.2.8]
    Seja $a = p_1 p_2 \dots p_k$ em que cada $p_i$ é um primo distinto.
    Vamos usar o Criterio de Eisenstein. 
    Seja p qualquer $p_i$ da decomposição de a.
    Temos então que $p \nmid 1$, $p \ | \ a$ pela hipotese sobre a, mas $p^2 \nmid a$, já que cada primo da decomposição de a é diferente.
    Logo $x^n - a$ é irredutivel.
\end{ex}

\begin{ex}[4.3.4]
    a) Suponha que G é um subgrupo de L que contém F, ou seja, $F \subset G \subset L$.
    Pelo Teorema de Extensão de Torres temos que
    \[ p = [L:F] = [L:G][G:F]\]
    Mas como p é primo, a unico possibilidade para esse produto é p e 1.
    Assim temos dois casos:
    \begin{itemize}
        \item Se $[L:G] = p$ e $[G:F] = 1$, então sabemos que $G = F$.
        \item Se $[L:G] = 1$ e $[G:F] = p$, então segue que $L = G$.
    \end{itemize}
    O que queriamos.
    b) Seja $\alpha \in L \setminus F$ e considere $F(\alpha)$. 
    Sabemos que $F \subset F(\alpha) \subset L$.
    Como vimos do item anterior, temos duas possibilidades, ou $F(\alpha) = F$ ou $F(\alpha) = L$.
    Mas $\alpha \notin F$, portanto $F(\alpha) \neq F$. Logo $F(\alpha) = L$.
\end{ex}

\begin{ex}[4.4.4]
    Seja $L | F$ uma extensão finita de grau n. Sabemos que se ela é finita então é algebrica.
    Tome então $\alpha \in L$ e considere $F(\alpha)$. 
    Sabemos que qualquer coleção de n+1 elementos em $F(\alpha)$ é L.D., então temos que 
    $1, \alpha, \dots, \alpha^n$ são L.D..
    Portanto existem $a_0, a_1, \dots, a_n \in F$ tais que 
    \[a_0 + a_1\alpha + \dots + a_n \alpha^n = 0\]
    Seja então $p(x) = a_0 + a_1 x + \dots + a_n x^n$. 
    Note então que $p(x) \in F[x]$ e $p(\alpha) = 0$, o que queriamos.
\end{ex}

\begin{ex}[5.1.4]
    As raizes da unidade do polinomio $f(x) = x^6 - 1$ são dadas por
    \[x = e^{2 \pi i k / 6} \quad k = 0, 1, 2, 3, 4, 5\]
    Então temos:
    \[1, \ e^{i \pi /3}, \ e^{2 i \pi /3}, \ -1, \ e^{4 i \pi /3}, \ e^{5 i \pi /3}\]
    Note que a primeira raiz complexa é:
    \[e^{i \pi /3} = \frac{1}{2} + i \frac{\sqrt{3}}{2}\]
    O corpo de decomposição de $f(x)$ deve conter todos as raízes. 
    As raizes 1 e -1 já estão em $\mathbb{Q}$, então precisamos de um corpo $\mathbb{Q}(\alpha)$ tal que 
    possue todas as demais raizes complexas.
    Note que a primeira raiz complexa pode ser escrita com $i\sqrt{3}$. 
    Além disso, as demais raizes complexas são geradas a partir da primeira, logo ${1, i\sqrt{3}}$ geram todas as raizes.
    Portanto o corpo de decomposição  de f é $\mathbb{Q}(i\sqrt{3})$.
\end{ex}

\begin{ex}[5.2.4]
    $\overline{\mathbb{Q}} \: | \: \mathbb{Q}$ é uma extensão normal que não é finita.
    Seja $p(x) \in \mathbb{Q}[x]$ o polinomio minimal de $\alpha \in \overline{\mathbb{Q}}$.
    Como $\overline{\mathbb{Q}}$ é algebricamente fechado, o corpo de decomposição de p está contido em 
    $\overline{\mathbb{Q}}$. Logo a extensão é normal.
    Para todo p primo, já vimos que $x^p -2 $ é irredutivel em $\mathbb{Q}$.
    Temos que $[\mathbb{Q}(x^p -2 ): \mathbb{Q}] = p$.
    Como temos infinitos primos e $\overline{\mathbb{Q}}$ contém todos os corpos de decomposição 
    dos polinomios dessa forma, segue que o grau da extensão é infinita entre 
    $\overline{\mathbb{Q}}$ e $\mathbb{Q}$. Portanto é normal mas não é finita.
\end{ex}

\begin{ex}[5.3.14]
    Tome $\alpha \in K$ um algebrico sobre F. 
    Como $K \subset L$, $\alpha \in L$ e portanto é algebrico sobre F.
    Sabemos que $L \: | \: F$ é separavel, então o polinomio $p_{\alpha \: | \: F}$ é separavel.
    Logo $K \: | \: F$ é separavel.
    Tome agora $\beta \in L$ algebrico sobre K.
    Queremos mostrar que o polinomio minimal de $\beta$ em K é separavel.
    Como $L \: | \: F$ é algebrico, $K \: | \: F$ é algebrico. 
    Logo $\beta$ é algebrico sobre F, portanto $p_{\beta \: | \: F}$ é separavel.
    Como $p_{\beta \: | \: K}$ divide $p_{\beta \: | \: F}$ e este é separavel, então $p_{\beta \: | \: k}$
    também deve ser separavel.
\end{ex}

\begin{ex}[5.4.7]
    Seja $K = F(\alpha_1, \dots, \alpha_{n-1})$. 
    Como cada $\alpha_i$ é separavel então $K \: | \: F$ é separavel.
    Note também que temos $L \: | \: K \: | \: F$, e como $L \: | \: F$ é finito, $K \: | \: F$ é também finito.
    Portanto pelo Teorema do Elemento Primitivo, $\exists \theta \in K$ tal que $K = F(\theta)$.
    Então agora temos $L = K(\alpha_n) = F(\theta)(\alpha_n)$.
    $\theta \in K$ é separavel sobre F já que $K \: | \: F$ é separavel.
    Como a extensão $FF(\theta)(\alpha_n)$ é finita, então $\exists \alpha \in K$ tal que 
    $L = F(\alpha)$. 
\end{ex}

\begin{ex}[6.1.6]
    Note que $\sqrt{6}$ e $\sqrt{10}$ geram $\sqrt{15}$, por:
    \[\frac{\sqrt{6}\sqrt{10}}{2} = \frac{\sqrt{60}}{2} = \frac{2\sqrt{15}}{2} = \sqrt{15}\]
    Então $\mathbb{Q}(\sqrt{6}, \sqrt{10}, \sqrt{15}) = \mathbb{Q}(\sqrt{6}, \sqrt{10})$.
    Calculamos agora o grau da extensão:
    O polinomio minimal de $\sqrt{6}$ e $\sqrt{10}$ são $x^2 - 6$ e $x^2 - 10$.
    Então 
    \[[\mathbb{Q}(\sqrt{6}, \sqrt{10}): \mathbb{Q}] = [\mathbb{Q}(\sqrt{6}, \sqrt{10}): \mathbb{Q}(\sqrt{10})][\mathbb{Q}(\sqrt{10}): \mathbb{Q}] = 2 \times 2 = 4\]
\end{ex}

\begin{ex}[6.2.3]
    a) O grau do polinomio $x^4 + x^3 + x^2 + x + 1$ é 4, então $[\mathbb{Q}(w): \mathbb{Q}] = 4$.
    O polinomio minimal de $\sqrt[5]{2}$ é $x^5 -2$, com grau 5 então $[\mathbb{Q}(\sqrt[5]{2}): \mathbb{Q}] = 5$.
    Como $w \notin \mathbb{R}$ então $w \notin \mathbb{Q}(\sqrt[5]{2})$, logo temos:
    \[[L : \mathbb{Q}] = [\mathbb{Q}(w, \sqrt[5]{2}): \mathbb{Q}(w)][\mathbb{Q}(w): \mathbb{Q}] = 5 \times 4 = 20\]
    b) Todas as raizes de $x^5 -2$ são $w\sqrt[5]{2}$ com $w = e^{2 \pi i / 5}$.
    Então $\mathbb{Q}(w, \sqrt[5]{2})$ é o corpo de decomposição.
\end{ex}

\begin{ex}[6.3.2]
    a) Os polinomios minimais de i e $\sqrt{2}$ são $x^2 + 1$ e $x^2 - 2$.
    Então o grau da extensão é 4, portanto $|Gal(\mathbb{Q}(i, \sqrt{2}) | \mathbb{Q})| = 4$.
    As raizes de $x^2 + 1$ são i e -i. 
    Já as raizes de $x^2 -2$ são $\sqrt{2}$ e $-\sqrt{2}$.
    Entao $\sigma(i) \in \{i, -i\}$ e fixa $\sqrt{2}$.
    E também $\sigma(\sqrt{2}) \in \{\sqrt{2}, -\sqrt{2}\}$ fixa i.
    Logo $\sigma^2 = Id$.
    Portanto temos o grupo de Klein, $V_4$.
    b) Já analisamos o caso para i. O polinomio minimal de $\sqrt[4]{2}$ é $x^4-2$. 
    Logo o grau da extensão é 8 e portanto $|Gal(\mathbb{Q}(i, \sqrt[4]{2}) | \mathbb{Q})| = 8$.
    Todas as raizes de $x^4-2$ são $\sqrt[4]{2}, -\sqrt[4]{2}, i\sqrt[4]{2}, -i\sqrt[4]{2}$.
    Analogo ao caso anterior temos $\{(1 \: 2 \: 3 \: 4), (2 \: 4)\} = D_4$.
\end{ex}

\end{document}