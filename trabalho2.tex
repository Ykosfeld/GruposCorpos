\documentclass{article}
\usepackage{amsthm}
\usepackage{amsmath}
\usepackage{amsfonts}
\usepackage{tikz}
\usepackage{amssymb}

\newtheorem*{definition}{Definição}
\newtheorem*{theorem}{Teorema}
\newtheorem*{exemple}{Exemplo}
\newtheorem*{atencao}{Atenção}
\newtheorem*{proposicao}{Proposição}
\newtheorem*{ex}{Exercício}
\usepackage{mathtools}
\usepackage{xfrac}
\DeclarePairedDelimiter\abs{\lvert}{\rvert}%
\DeclareMathOperator{\Ima}{Im}

\setlength{\textwidth}{450pt}
\setlength{\marginparwidth}{0pt}
\setlength{\marginparsep}{0pt}
\usepackage[left=2.5cm, right=2.5cm]{geometry}

\title{Trabalho 2 \\ \large Grupos e Corpos}
\author{Yuri Kosfeld}
\date{Junho 2025}

\begin{document}

\maketitle

\begin{ex}[4.1.8]
    a) Poderiámos provar utilizando Eisenstein, mas vamos mostrar por um processo analogo a um exemplo dessa seção.
    Informalmente, sabemos que uma raiz de $x^2 -3$ é $\sqrt{3}$, assim sabemos que o polinomio pode ser reescrito como 
    $x^2 -3 = (x - \sqrt{3})(x + \sqrt{3})$. Então se $\sqrt{3} \in \mathbb{Q}(\sqrt{2})$, o polinomio é redutivel.
    Suponha então que $\sqrt{3} \in \mathbb{Q}(\sqrt{2})$. Logo $\exists a, b \in \mathbb{Q}$ tais que $\sqrt{3} = a + b\sqrt{2}$.
    \begin{align*}
        (\sqrt{3})^2 &= (a + b\sqrt{2})^2 \\
        3 &= a^2 + 2ab\sqrt{2} + 2b^2 \\
        &= (a^2 + 2b^2) + (2ab)\sqrt{2}\\
    \end{align*}
    Assim temos, $a^2 + 2b^2 = 3$ e $2ab = 0$.
    Como $2ab = 0$, então ou $a = 0$ ou $b = 0$.
    \begin{itemize}
        \item Se $a = 0$, então $2b^2 = 3 \Rightarrow b = \sqrt{3/2}$ e logo $b \notin \mathbb{Q}$, absurdo.
        \item O caso $b = 0$ é analago ao anterior, também chegando em um absurdo. 
    \end{itemize}
    Logo $\sqrt{3} \notin \mathbb{Q}(\sqrt{2})$ e assim o polinomio é irredutivel.
\end{ex}

\end{document}