\documentclass[a4paper,12pt]{article}
\usepackage{graphicx} % Required for inserting images

\title{\Large \bfseries Seminário 2 - Notas\\Simplicidade de An, para $n \geq 5$}
\author{Davi Bossatto \\ Enzo Lavezzo \\ Yuri Florentino \\}
\date{}

\usepackage[utf8]{inputenc}
\usepackage[T1]{fontenc}
\usepackage[portuguese]{babel}
\usepackage[]{amsthm} %lets us use \begin{proof}
\usepackage[]{amssymb} %gives us the character \varnothing
\usepackage{amsmath, amsfonts, amssymb}
\usepackage{xcolor}
\usepackage{csquotes}

\newtheorem*{theorem}{Teorema}
\newtheorem*{proposition}{Proposição}
\newtheorem*{lemma}{Lema}
\newtheorem*{corollary}{Corolário}
\theoremstyle{definition}
\newtheorem{definition}{Definição}
\theoremstyle{remark}
\newtheorem*{remark}{Observação}
\newtheorem*{example}{Exemplo}

\usepackage[top=1.5cm, bottom=2cm, left=1.5cm, right=1.5cm]{geometry}
\setlength {\marginparwidth }{1cm}

\begin{document}

\maketitle

\section{$A_n$ e as equações de quinto grau}

Agora que vimos a Simplicidade dos grupos alternados de ordem maior ou igual que cinco, estamos mais próximos de entender porque equações de quinto grau ou mais não possuem uma 
fórmula explicita para suas raizes. 

Precisamos antes, relembrar algumas definições e provar agluns resultados vistos em aula.

\begin{definition}
    Seja G um grupo. Uma sequencia de subgrupos
    \[H^{\bullet}: \{e\} \leq H_1 \leq H_2 \leq \dots \leq H_n = G\]
    é chamada série subnormal se $H_i \vartriangleleft H_{i+1}$. 
    Os grupos $H_i/H_{i+1}$ são chamados de fatores da série.
\end{definition}

\begin{definition}
    Um grupo G é solúvel se possui uma série subnormal, tal que todo grupo fator é abeliano.
\end{definition}

\begin{definition}
    Dado um grupo G, definimos o grupo derivado de G por $G' = [G, G]$.
    De modo mais geral definimos, $G^{(0)} = G$ e $G^{(i+1)} = [G^{(i)}, G^{(i)}]$. 
\end{definition}

\begin{proposition}
    Sejam G um grupo e G' o seu grupo derivado. Então valem:
    \begin{enumerate}
        \item $G/G'$ é abeliano;
        \item $G'$ é o menor subgrupo normal de G tal que se $H \lhd G$ com $G/H$ abeliano, então $G' \subseteq  H$.
    \end{enumerate}
\end{proposition}

\begin{proof}
    Sejam $aG', bG' \in G/G'$. Para mostrar que $G/G'$ é abeliano, precisamos verificar que:
    \[abG' = baG'\]
    Isso é equivalente a mostrar que $a^{-1}b^{-1}ab \in G'$. Mas note que $[a, b] = a^{-1}b^{-1}ab$ é um comutador,
    logo por definição pertence a $G'$.
    Suponha agora que $H \vartriangleleft G$ é um subgrupo normal tal que $G/H$ é abeliano. Queremos mostrar que $G' \subseteq H$.
    Se $G/H$ é abeliano, então como vimos anteriormente, $ a^{-1}b^{-1}ab \in H$, ou seja, $[a, b] \in H$.
    Como $G'$ é gerado por todos os comutadores $[a, b]$ e todos eles pertencem a H, segue que $G' \subseteq H$, uma vez que, 
    H é o grupo que contém todos os geradores de $G'$.
\end{proof}

Com esses resultados, podemos agora demonstrar o seguinte teorema

\begin{theorem}
    Seja G um grupo. Então G é solúvel se e somente se existe $n \in \mathbb{N}$ tal que $G^{(n)} = \{e_G\}$. 
\end{theorem}

\begin{proof}
    $\Rightarrow $: Suponha que G é solúvel, e considere a seguinte série subnormal
    \[H^{\bullet}: \{e\} \leq H_1 \leq \dots \leq H_n = G\]
    com todos os grupos fatores abelianos.
    Como temos que $H_n/H_{n-1}$ é abeliano, do resultado anterior segue que $G'\subseteq H_{n-1}$.
    Novamente pela proposição, $H_{n-1}/H_{n-2}$ é abeliano e então $H_{n-1}' \subseteq H_{n-2}$.
    Mas $G' \subseteq H_{n-1}$ , donde segue que $G^{(2)} \subseteq H_{n-1}' \subseteq H_{n-2}$.
    Se repetirmos o mesmo processo para os demais grupos fatores, chegamos em $G^{(n)} \subseteq \{e\}$, o que queriamos.
    $\Leftarrow $: Suponha agora que exista $n \in \mathbb{N}$ tal que $G^{(n)} \subseteq \{e\}$.
    Temos então que a série de G: $H^{\bullet}: \{e\} \leq H_1 \leq \dots \leq H_n = G$ é subnormal, cujo grupos fatores são todos abelianos.
    Logo G é solúvel.
\end{proof}

\begin{theorem}
    Sejam G um grupo e H um subgrupo normal de G. Então G é solúvel se e somente se H e $G/H$ são também soluveis.
\end{theorem}

\begin{proof}
    Seja $\varphi: G \rightarrow G/H$ o homomorfismo canônico. Vamos primeiro mostar que $\varphi(G') = (\varphi(G))'$.
    Seja $[x, y] \in G'$ um comutador de G. Aplicando $\varphi$ temos
    \[\varphi([x, y]) = \varphi(xyx^{-1}y^{-1}) = (xyx^{-1}y^{-1})H = (xH)(yH)(x^{-1}H)(y^{-1}H) = [xH, yH] \in (G/H)'\]
    Logo $f(G') \subseteq (G/H)'$.
    Tome agora $[\overline{x}, \overline{y}] \in (G/H)'$, em que $\overline{x} = xH$ e $\overline{y} = yH$.
    Temos então
    \[[\overline{x}, \overline{y}] = \overline{x}\overline{y}\overline{x}^{-1}\overline{y}^{-1} = (xH)(yH)(x^{-1}H)(y^{-1}H) = (xyx^{-1}y^{-1})H = \varphi([x, y])\]
    Assim, $(G/H)' \subseteq f(G')$, e então $\varphi(G') = (\varphi(G))'$.
    Segue então que 
    \[\varphi(G^{(2)}) = \varphi((G')')=(\varphi(G'))'=((\varphi(G)'))'=(\varphi(G))^{(2)}\]
    e indutivamente temos $\varphi(G^{(i)}) = (\varphi(G))^{(i)} = (G/H)^{(i)}$ para todo $i \geq 0$.
    Suponha então que G é solúvel e seja n um inteiro tal que $G^{(n)} = \{e\}$, segue que $H^{(n)} = \{e\}$
    e $(G/H)^{(n)}=\varphi(G^{(n)})=\varphi(\{e\})=\{e\}$. Ou seja, os grupos H e $G/H$ são soluveis. 
    Reciprocamente, suponha que H e $G/H$ são soluveis e sejam n e m inteiros tais que $H^{(n)}=\{e\}$ e $(G/H)^{(m)}=\{e\}$.
    Como $\{e\}=(G/H)^{(m)}=\varphi(G^{(m)})$, temos $G^{(m)} \subseteq \ker \varphi = H$.
    Logo $G^{(n+m)} = (G^{(m)})^{(n)} \subseteq H^{(n)} = \{e\}$, e assim ganhamos que G é solúvel.
\end{proof}

Por fim, vamos definir mais um objeto de estudo e enunciar um útilmo teorema.

\begin{definition}
    Um polinômio $p(x) \in K[x]$, onde $K$ é um corpo (como $\mathbb{Q}$), é dito expresso por radicais ou resolúvel por radicais
    se todas suas raízes podem ser obtidas a partir de elementos de $K$ usando uma combinação finita das operações aritmeticas e raizes n-esimas.
\end{definition}

\begin{theorem}
    As raizes de um polinômio podem ser expressa por radicais se e somente se o grupo de permutações do polinômio é solúvel.
\end{theorem}

Finalmente, sabemos que o grupo de permutações de um polinômio de grau maior ou igual 5 é o $S_n$. Como vimos pelo Corolário 3, $A_n$ e $S_n$ são os 
únicos subgrupos normais de $S_n$. Mas para $n \geq 5$, $A_n$ é simples e portanto não é abeliano e donde segue que $A_n$ não é solúvel.
Logo pelo teorema anterior $S_n$ não é solúvel, e pelo teorema de Galois, o polinômio de grau maior ou igual a 5 não pode ser escrito 
por radicais. Isso equivale a dizer que não temos uma fórmula geral para esses polinômios.

\end{document} 